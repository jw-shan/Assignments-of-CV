\documentclass[10 pt,hyperref={colorlinks = true,linkcolor = blue}]{beamer}
\usepackage{color,fancybox,bm}
\usepackage{color}
\usepackage{booktabs}
\usepackage{threeparttable}
\usepackage{dashrule}
\usepackage{float}
\usepackage{graphicx}
\usepackage{amsmath}
\usepackage{fixltx2e}
\usepackage{amssymb}
\usepackage{rotating}
\usepackage{beamerthemeshadow}
\newtheorem{acknowledgement}[theorem]{Acknowledgement}
\newtheorem{algorithm}[theorem]{Algorithm}
\newtheorem{assumption}{Assumption}
\newtheorem{assumption1}{Assumption 1}
\newtheorem{assumption2}{Assumption 2}
\newtheorem{assumptions}{Assumptions}
\newtheorem{axiom}[theorem]{Axiom}
\newtheorem{case}[theorem]{Case}
\newtheorem{thm}[theorem]{Theorem}
\newtheorem{thm2}[theorem]{Theorem 2}
\newtheorem{thm3}[theorem]{Theorem 3}
\newtheorem{claim}[theorem]{Claim}
\newtheorem{lemma1}[theorem]{Lemma 1}
\newtheorem{lemma2}[theorem]{Lemma 2}
\newtheorem{lemma4}[theorem]{Lemma 4}
\newtheorem{coro5}[theorem]{Corollary 5}
\newtheorem{coro7}[theorem]{Corollary 7}
\newtheorem{coro9}[theorem]{Corollary 9}
\newtheorem{coro11}[theorem]{Corollary 11}
\newtheorem{conclusion}[theorem]{Conclusion}
\newtheorem{condition}[theorem]{Condition}
\newtheorem{conjecture}[theorem]{Conjecture}
\newtheorem{criterion}[theorem]{Criterion}
\newtheorem{exercise}[theorem]{Exercise}
\newtheorem{notation}[theorem]{Notation}
\newtheorem{proposition}[theorem]{Proposition}
\newtheorem{remark}[theorem]{Remark}
\newtheorem{summary}[theorem]{Summary}
\newcommand{\nc}{\newcommand}
\nc{\tr}{\text{tr}}



\useoutertheme{infolines}
\setbeamercolor*{palette
primary}{use=structure,fg=structure.fg,bg=structure.fg!40!white}
\setbeamercolor*{palette
secondary}{use=structure,fg=white,bg=structure.fg!60!white}
\setbeamercolor*{palette
tertiary}{use=structure,fg=white,bg=structure.fg!90!white}
\setbeamercolor*{palette quaternary}{fg=white,bg=black}

\setbeamercolor*{sidebar}{use=structure,bg=structure.fg}

\setbeamercolor*{palette sidebar
primary}{use=structure,fg=structure.fg!10} \setbeamercolor*{palette
sidebar secondary}{fg=white} \setbeamercolor*{palette sidebar
tertiary}{use=structure,fg=structure.fg!50} \setbeamercolor*{palette
sidebar quaternary}{fg=white}

\setbeamercolor*{titlelike}{use=structure,fg=structure.fg,bg=structure.fg!20!white}

\setbeamercolor*{separation line}{} \setbeamercolor*{fine separation
line}{}

\setbeamercolor*{block title example}{fg=black}




\usefonttheme[onlysmall]{structurebold}

\newcommand{\ft}{\frametitle}
\newcommand{\bb}{\begin{block}}
\newcommand{\eb}{\end{block}}
\newcommand{\bi}{\begin{itemize}}
\newcommand{\ei}{\end{itemize}}
\newcommand{\be}{\begin{enumerate}}
\newcommand{\ee}{\end{enumerate}}
\newcommand{\bab}{\begin{alertblock}}
\newcommand{\eab}{\end{alertblock}}
\newcommand{\beb}{\begin{exampleblock}}
\newcommand{\eeb}{\end{exampleblock}}
\newcommand{\bc}{\begin{columns}}
\newcommand{\ec}{\end{columns}}
\newcommand{\ii}{\item}
\newcommand{\convas}{\stackrel{a.s.}{\rightarrow}}
\newcommand{\convp}{\stackrel{p}{\rightarrow}}
\newcommand{\convd}{\stackrel{d}{\rightarrow}}
\newcommand{\ba}{\begin{array}}
\newcommand{\ea}{\end{array}}
\newcommand{\bl}{\color{blue}}
\usepackage[backend=bibtex,sorting=none]{biblatex}



\title[ ]{Depix——a tool for recovering passwords from pixelized screenshots
}


\author[Jiawei Shan]{\small{Github Page: \url{github.com/beurtschipper/Depix}}\\
~\\
Reporter: Jiawei Shan \\[2mm]}
\institute[]{Institute of Statistics \& Big Data \\ Renmin University of China

}

\date[December 30, 2020]{\footnotesize{December 30, 2020}}

\begin{document}

	\begin{frame}
	\titlepage
\end{frame}

% \section{Introduction}

% =========================================
\begin{frame}{Introduction}

{\bl What is Pixelization?}

\pause

\begin{figure}
	\includegraphics[scale=0.23]{1.jpg}
\end{figure}

\end{frame}


% =========================================
\begin{frame}{Introduction}

	{\bl What is Pixelization?}

	\begin{figure}
		\includegraphics[scale=0.25]{2.jpg}
	\end{figure}
\end{frame}

% =========================================
\begin{frame}{Introduction}

\begin{itemize}
	\item Pixelization describes the process of partially lowering the resolution of an image to censor information.
	\item The \textbf{linear box filter} method is  commonly used to implement pixelization, which is simple and works fast.
	\item A linear box filter takes a box of pixels, and overwrites the pixels with the average value of all pixels in the box.
\end{itemize}

~

Here is an illustation of the linear box filter:

\begin{figure}
	\includegraphics[scale=1.2]{example.png}
	\caption{Illustration of the linear box filter.}
\end{figure}

\end{frame}

% =========================================
\begin{frame}{Introduction}

Depix is a tool for recovering passwords from pixelized screenshots, and the image below shows one of the test results.
\begin{figure}
	\includegraphics[scale=0.23]{result.png}
	\caption{Test result of Depix.}
\end{figure}

\end{frame}


% ===================================
\begin{frame}{Algorithm description}
\begin{itemize}
	\item Since the linear box filter is a deterministic algorithm, pizelizing the same values will always result in the same pixelated block.
	\item Pixelizing the same text - using the same locations of blocks - will result in the same block values.
	\item We can try to pixelate text to find matching patterns.
	\item This solution is quite simple: take a \textbf{De Bruijn sequence} of expected characters, paste it in the same editor, and make a screenshot of that. That screenshot is used as a lookup image for similar blocks.
\end{itemize}

\end{frame}

% ===================================
\begin{frame}{Algorithm description}

{\bl De Bruijn sequence }

\begin{itemize}
	\item In combinatorial mathematics, a \textbf{De Bruijn sequence} of order $n$ on a size-$k$ alphabet $A$, denoted by $B(k,n)$ is a cyclic sequence in which every possible length-$n$ string on $A$ occurs exactly once as a substring (i.e., as a contiguous subsequence).
	\item For example, taking $ A = \{0, 1\}$, there are two distinct $B(2, 3)$: $00010111$ and  $11101000$, one being the reverse or negation of the other.
\end{itemize}

\end{frame}


% ===================================
\begin{frame}{Algorithm description}
\begin{itemize}
	\item \href{run:debruinseq.txt}{Here} is an example of De Bruijn sequence with $n=2$ and $A=\{0-9\}\bigcup\{a-z\}\bigcup\{A-Z\}$.
	\item It's important that 2-character combinations are used, because some blocks can overlap two characters.
\end{itemize}

~

\begin{figure}
	\includegraphics[scale=0.23]{search.png}
	\caption{Search using De Bruijn sequence.}
\end{figure}
\end{frame}



% ======================================
\begin{frame}
Finally, we show a test result in \href{http://localhost:8886}{jupyter notebook}\footnote{\href{run:jupyter.cmd}{ssh-connection}}.



\end{frame}



\end{document}
